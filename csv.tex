\documentclass[11pt]{article}
\usepackage[utf8]{inputenc}

\renewcommand{\thesection}{Article \arabic{section}}
\setlength{\parindent}{0cm}

\usepackage{geometry}
\geometry{letterpaper,tmargin=2cm,bmargin=2cm,lmargin=3cm,rmargin=3cm}

\begin{document}

\begin{center}
{\Large Conditions Générales de Vente}
\end{center}

\hrule height .1mm

\vspace{.5cm}

\noindent Ce contrat applicable à compter du 1er janvier 2014 entre les soussignés :
\\
\\
{\bf FullStack EURL au capital social de 5000 euros\\
immatriculée au RCS de Paris sous le numéro ______ \\
dont le siège est sis 3B rue Jean-Marie Jégo 75013 Paris\\
représentée aux fins des présentes par Sébastien Saunier en qualité de gérant
}
\\
\\
(ci-après dénommée ``FullStack''), and\\
\\
{\bf {{ client.nom }} au capital social de {( client.capital )} euros\\
immatriculée au RCS de {{ client.rcs_ville }} sous le numéro {{ client.rcs_numero }}\\
dont le siège social est sis {{ client.adresse }}\\
représentée aux fins des présentes par {{ client.representant_nom }} en qualité de {{ client.representant_qualite }}
}
\\
\\
(ci-après le ``Client'').


\section{Documents contractuels}

Les documents contractuels, par ordre décroissant de priorité, sont les suivants :

\begin{enumerate}
  \item Conditions particulières
  \item Backlogs
  \item Icebox
  \item Conditions générales de vente
\end{enumerate}

Ils constituent l’intégralité des obligations des parties. En cas de contradiction entre les différents éléments, le document le mieux classé dans l’ordre de priorité prévaudra.

\section{Définitions}

\begin{itemize}
  \item \textbf{Itération} Cycle de développement aux termes duquel une version fonctionnelle du Livrable est livrée par FullStack au Client.
  \item \textbf{Livrable} Version fonctionnelle d’un site web, logiciel, éléments graphiques, éditoriaux ou autres, livrés par FullStack au Client à l’issue de chaque Itération.
  \item \textbf{Icebox} spécifications fonctionnelles définies par le Client, sous la forme de cas simples et factuels.
  \item \textbf{Backlog} document contractuel par lequel FullStack reformule l’Icebox en termes de fonctionnalités techniques auxquels un indice de difficulté est associé.
  \item \textbf{Correction} Toute correction d’erreur sur le Livrable, n’ayant pas d’autre finalité que de corriger des dysfonctionnements ou des erreurs.
  \item \textbf{Évolution} Toute modification, évolution, adaptation, adjonction, suppression, traduction, nouvelle version, ne constituant pas une simple Correction.
  \item \textbf{Élément apporté} contenu de toute nature (codes, textes, images, sons…) inséré dans le Livrable, soit par FullStack à la demande du Client, soit par le Client lui-même.
\end{itemize}


\section{Objet du contrat}

Par le présent contrat, FullStack s’engage à concevoir et à réaliser pour le compte du Client un Livrable conformément à la méthode Agile.

\section{Obligations des parties}

\subsection{Obligations de FullStack}

FullStack s’engage à concevoir et à réaliser un Livrable fonctionnel par Itérations successives. Il devra ainsi apporter les solutions techniques permettant de concrétiser l’architecture du Livrable et d’en assurer le bon fonctionnement. FullStack est soumis à une \textbf{obligation de moyens}. FullStack pourra sous-traiter tout ou partie de la prestation. Sauf stipulation contraire dans les Conditions particulières, FullStack n’est soumis à aucune obligation de maintenance ou d’assistance à l’utilisation.

\subsection{Obligations du Client}

Le Client s’engage à fournir à FullStack les Éléments apportés. Il est précisé à ce titre, sauf mention contraire dans les Conditions particulières, que la rédaction des textes et la saisie du contenu administrable est à la charge exclusive du Client. Le Client s’engage à réagir dans un délai de 3 jours à partir de la mise en demeure au plus tard à toute demande de FullStack relative à la réalisation du Livrable. Le Client reconnaît que son implication et sa collaboration sont nécessaires pour la bonne exécution du contrat. Tout retard ou défaut de diligence du Client pourra entraîner un report de livraison de FullStack supérieur d’une semaine à la durée dudit retard ou défaut de diligence, sans que ce report ne puisse constituer une faute de FullStack ou un préjudice pour le Client.

\section{Méthodes de travail}

Les parties déterminent dans les Conditions particulières la méthode de travail parmi :

\subsection{Développements itératifs}

Dans le cadre de cette méthode de travail, FullStack s’engage à concevoir un Livrable conformément à la méthode Agile. Le développement du Livrable fait en conséquence l’objet de plusieurs Itérations. Chaque Itération est soumise au plan de travail suivant :

\begin{itemize}
  \item Les parties conviennent de la durée de l’Itération (1, 5 ou 10 jours ouvrés)
  \item Le Client définit les spécifications fonctionnelles dans un document appelé Icebox
  \item FullStack exprime ces spécifications fonctionnelles en terme de fonctionnalités techniques auxquels est associé un indice de difficulté, dans un document appelé Backlog.
  \item Le Client détermine les priorités parmi les fonctionnalités techniques.
  \item Le cycle de développement peut commencer lorsque les parties considèrent que la charge de travail est suffisante pour l’Itération.
  \item Lors d’un bref entretien quotidien (environ 10 minutes), les parties font le point sur l’état d’avancement et les problèmes rencontrés.
  \item Aux termes de l’Itération, le Livrable est livré au Client et réceptionné conformément à l’article 6 des présentes.
  \item À l’issue de l’Itération, un bilan est effectué entre les parties, afin de savoir si une nouvelle Itération est nécessaire pour la mise en production du Livrable.
\end{itemize}


\subsection{Développement planifié}

Dans ce cadre, FullStack s’engage à concevoir un Livrable selon la méthode classique de développement, dite Waterfall. Le développement complet du Livrable se déroulera alors en deux étapes.

\subsubsection{Étape 1 – Conception du Livrable}

Suite à la remise par le Client des informations nécessaires à la réalisation du Livrable, FullStack établira l’architecture du Livrable, en tenant compte, le cas échéant, du cahier des charges fourni par le Client.

\subsubsection{Étape 2 – Réalisation du Livrable}

Suite à la validation de l’architecture par le Client, FullStack procédera à la réalisation du Livrable. Au terme de la phase de développement, le Livrable est livré au Client et réceptionné conformément à l’article 6 des présentes.


\subsection{Méthode ad hoc}

Les parties pourront convenir d’une méthode de travail ad hoc. Dans ce cas, la méthode de travail envisagée fera l’objet d’une description dans les Conditions particulières.


\section{Procédure de réception}

À la fin de chacune des phases de développement, FullStack livre au Client un Livrable. Dans un délai de 1 semaine à compter de la livraison, le Client valide le Livrable ou émet des réserves sur celui-ci. Si le Client émet des réserves dans le délai précité, FullStack procédera aux Corrections du Livrable, à l’exclusion de toute Évolution. Si le Client n’a émis aucune réserve dans le délai précité, le Livrable soumis au Client est considéré comme étant tacitement réceptionnée par celui-ci.


\section{Responsabilité}

\subsection{Limitation de responsabilité}

FullStack n’est pas responsable des dommages que l’installation du Livrable pourrait causer aux serveurs et matériels informatiques du Client. Dans l’hypothèse où le Client réaliserait lui-même une Évolution ou en confierait la réalisation à un tiers, FullStack ne sera plus en mesure de prendre en charge les Corrections du Livrable. FullStack exclut par ailleurs toute responsabilité au titre des Évolutions apportées par le Client ou un tiers au Livrable. La réception du Livrable par le Client couvre les défauts de conformité apparents et les vices apparents. Toute action à l’encontre de FullStack, notamment au titre d’un défaut de conformité caché ou d’un vice caché, est prescrite dans le délai d’un an à compter de la réception définitive du Livrable par le Client. FullStack exclut toute responsabilité au titre de la réservation ou de l’utilisation d’un signe distinctif (nom de domaine, marque, titre, nom commercial, dénomination sociale...) à laquelle le Client lui aurait demandé de procéder.

La responsabilité de FullStack est limitée au préjudice direct, personnel et certain subi par le Client et lié à la défaillance en cause. FullStack ne pourra en aucun cas être tenue responsable des dommages indirects tels que, notamment, les pertes de données, les préjudices commerciaux, les pertes de commandes, les atteintes à l’image de marque, les troubles commerciaux et les pertes de bénéfices ou de clients. Le montant des dommages et intérêts mis à la charge de FullStack ne pourra excéder le montant du prix total du Livrable, tel qu’indiqué dans les Conditions particulières.

Aux termes de chaque Itération, FullStack conserve son entière liberté de poursuivre ou de ne pas poursuivre le projet avec le Client. FullStack pourra en conséquence arrêter les développements après chaque livraison d’une Itération, sans que cela ne constitue une faute de FullStack ou un préjudice pour le Client.

\subsection{Force majeure et faute du client}

FullStack n’engagera pas sa responsabilité en cas de force majeure ou de faute du Client, telles que définies ci-après :

Sera considéré comme un cas de force majeure opposable au Client tout empêchement, limitation ou dérangement du fait d’incendie, d’épidémie, d’explosion, de tremblement de terre, de fluctuations de la bande passante, de manquement imputable au fournisseur d’accès, de défaillance des réseaux de transmission, d’effondrement des installations, d’utilisation illicite ou frauduleuse des mots de passe, codes ou références fournis au Client, de piratage informatique, d’une faille de sécurité imputable à l’hébergeur du Site, d’inondation, de panne d’électricité, de guerre, d’embargo, de loi, d’injonction, de demande ou d’exigence de tout gouvernement, de réquisition, de grève, de boycott, ou autres circonstances hors du contrôle raisonnable de FullStack.

Sera considérée comme une faute du Client opposable à ce dernier toute mauvaise utilisation du Livrable, faute, négligence, omission ou défaillance de sa part ou de celle de ses préposés, non-respect des conseils donnés par FullStack.


\section{Garanties}

\subsection{Garanties de FullStack}

FullStack garantit au Client que le Livrable sera achevé et livré conformément aux documents contractuels. À défaut de mention contraire dans les Conditions particulières concernant la licence de droit de propriété intellectuelle, FullStack ne donne pas d’autre garantie que celle de l’éviction de son fait personnel et de l’existence matérielle des droits cédés.

\subsection{Garanties du Client}

Le Client déclare être titulaire de tous les droits et autorisations permettant l’utilisation des Éléments apportés aux fins des présentes. Il garantit FullStack contre toute action et revendication de tiers du fait des Éléments apportés.

\section{Propriété intellectuelle}

\subsection{Titularité des droits sur le Livrable}

Le Livrable est composé le cas échéant :

\begin{itemize}
\item d’un framework : composant logiciel structurel, définissant les fondations du Livrable. Le framework est placé sous licence spécifique. Le Client est investi des droits sur le framework conformément à cette licence spécifique.

\item des développements spécifiques : contenus réalisés spécifiquement par FullStack pour le Client dans le cadre de la méthode Agile. Le Client est investi des droits sur les développements spécifiques conformément à l’article 9.2 des présentes conditions générales

\item des Éléments apportés : contenus de toute nature (codes, textes, images, sons…) insérés dans le Livrable, soit par FullStack à la demande du Client, soit par le Client lui-même. Le Client reste titulaire des droits sur les Éléments apportés qu’il met à la disposition de FullStack pour la réalisation de la prestation.
\end{itemize}

\subsection{Licence exclusive sur le Livrable}

À défaut de mention contraire dans les Conditions particulières, FullStack consent au Client à titre exclusif une licence portant sur l’intégralité des droits patrimoniaux sur les développements spécifiques, et notamment sur les droits d’exploitation, de reproduction, de représentation, d’édition, de commercialisation, de traduction dont il est titulaire, pour toute la durée de protection des droits de propriété intellectuelle et pour le monde entier, par tout procédé, quel qu’il soit, connu ou inconnu à ce jour, et notamment par tous les réseaux de communication, actuels et futurs, et ce sur tout support, en tout format.

S’il s’agit de logiciel, la présente licence porte sur le code objet, le code source et la documentation du logiciel, FullStack autorisant le Client à accéder aux codes du logiciel.

À défaut de mention contraire dans les Conditions particulières, le Client ne pourra pas consentir de sous-licences sur les développements spécifiques.

\subsection{Rémunération au titre de la licence}

La rémunération de FullStack au titre de la licence des droits d’auteur sur les développements spécifiques est forfaitaire.

FullStack ne pourra prétendre au titre de sa prestation et de la licence de droit d’auteur sur les développements spécifiques à d’autres rémunérations que celles stipulées dans les Conditions particulières.

\subsection{Droit de paternité}

À défaut de mention contraire dans les conditions particulières, le nom de Sébastien Saunier, auteur originaire du Logiciel, devra être associé au Livrable au cours de son exploitation. S’il s’agit d’un site web, le nom de Sébastien Saunier apparaîtra :

\begin{itemize}
  \item sur une page appelé ``Crédit'', ``Informations légales'' ou ``Mentions légales'' accessible depuis la page d’accueil du site et comportant un lien hypertexte vers www.fullstack.fr
  \item dans le header du code html des pages web du site sous la balise ``Creator''.
\end{itemize}

\subsection{Actions en justice}

À défaut de mention contraire dans les conditions particulières, seule FullStack aura qualité pour exercer les actions en contrefaçon et concurrence déloyale relatives aux exploitations non autorisées des développements spécifiques.

Toutefois, lorsque FullStack aura autorisé le Client à consentir des sous-licences sur les développements spécifiques, seul le Client aura qualité pour exercer les actions en contrefaçon et concurrence déloyale relatives aux exploitations non autorisées des développements spécifiques.

\section{Evolution - Maintenance}

FullStack autorise le Client à effectuer lui-même des Évolutions du Livrable et à en assurer la maintenance.

Toutefois, si le Client souhaite confier les Évolutions ou la maintenance du Livrable à un prestataire, le Client s’engage à proposer en priorité la prise en charge de cette prestation à FullStack, qui s’engage en contrepartie à proposer au Client des conditions tarifaires du même ordre de grandeur à celles consenties aux termes du présent contrat.


\section{Prix et modalités de paiement}

\subsection{Prix}

Le taux horaire ou journalier est celui convenu dans les Conditions particulières.
À défaut de stipulation contraire dans les Conditions particulières, le paiement interviendra avant le commencement de chacune des Itérations.

\subsection{Rabais, Remise, Ristourne}

Sauf accord exprès des Parties, aucun rabais, remise, ristourne ou escompte ne sera accordé.

\subsection{Intérêts en cas de retard}
En cas de non-paiement à son échéance, toute somme due portera automatiquement intérêt conformément aux dispositions légales en vigueur.

\subsection{Rémunération supplémentaire}
En cas de demandes de la part du Client entraînant une charge de travail supplémentaire pour FullStack, les parties conviendront, avant toute modification du Plan de travail ou du Cahier des charges, du versement d’une somme supplémentaire au profit de FullStack.

\section{Références}
Sauf interdiction expresse du Client dans un délai d’un mois à compter de la réception du Livrable, FullStack est autorisé à utiliser le nom du Client, ainsi que l’image de la page d’accueil du Livrable, dans le cadre de sa promotion commerciale et publicitaire.

\section{Réserve de propriété}
FullStack se réserve expressément la propriété du Livrable jusqu’au paiement intégral de son prix en principal et intérêts. Le Client en deviendra toutefois responsable dès sa livraison, le transfert de possession entraînant celui des risques.

\section{Non-sollicitation}
Afin de garantir l’équilibre économique du Contrat, le Client s’engage à ne pas embaucher, tenter d’embaucher ou faire travailler directement ou indirectement un salarié, un collaborateur ou un sous-traitant de FullStack durant l’exécution du présent contrat et jusqu’à 1 an après la réception du Livrable.

En cas de non-respect de cette clause, le Client devra verser à FullStack à titre de clause pénale, et sans préjudice des éventuels dommages-intérêts auxquels pourrait prétendre FullStack, une somme égale à la rémunération de FullStack stipulée dans les Conditions particulières.

\section{Résiliation}
En cas de manquement grave de l’une ou l’autre des parties aux obligations du présent contrat non réparé dans un délai de 15 jours à compter de la présentation de la lettre recommandée avec accusé de réception notifiant ledit manquement, le présent contrat sera résilié de plein droit sans préjudice des dommages et intérêts auxquels les parties pourraient prétendre.

Le fait pour une des parties de ne pas se prévaloir d’un manquement par l’autre partie à l’une quelconque des obligations visées dans les présentes ne saurait être interprété pour l’avenir comme une renonciation à l’obligation en cause.

\section{Loi applicable et juridiction compétente}
Le présent contrat est soumis à la loi française. En cas de litige, la compétence exclusive est attribuée aux tribunaux de Paris, même pour les procédures d’urgence ou conservatoire en référé ou par requête, nonobstant pluralité de défendeurs ou appel en garantie.


\vspace{1cm}

\noindent The undersigned agrees to the terms of this agreement on behalf of his or
her organization or business.\\\\

\noindent \begin{tabular}{l l l}
On behalf of the Client: & \rule{6cm}{.2pt} & Date: \rule{2.4cm}{.2pt}\\
                         & CLIENT REPRESENTATIVE      & \\\\\\
The Consultant:          & \rule{6cm}{.2pt} & Date: \rule{2.4cm}{.2pt}\\
                         & CONSULTANT NAME      & \\
\end{tabular}

\end{document}
